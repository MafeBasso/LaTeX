\documentclass{article}
 
%encoding
%--------------------------------------
\usepackage[T1]{fontenc}
\usepackage[utf8]{inputenc}
%--------------------------------------
 
%Portuguese-specific commands
%--------------------------------------
\usepackage[portuguese]{babel}
%--------------------------------------
 
%Hyphenation rules
%--------------------------------------
\usepackage{hyphenat}
\hyphenation{mate-mática recu-perar}
%--------------------------------------

\begin{document}

\textbf{Trabalho de Cálculo 2}

Para cada n inteiro não negativo seja \(I_n\) definido como

\[I_n = \int_{0}^{\pi/2} cos^{2n}x \ dx\]

Em particular, tem-se \(I_0 = \pi/2\)

\textbf{31.3.1 Exercício:}

\textbf{(a)} Integrando por partes verifique a seguinte fórmula de redução:
\[I_n = \int cos^{2n}x \ dx = \frac{senx \ cos^{2n-1}x}{2n} + \frac{2n-1}{2n} \ \int cos^{2n-2}x \ dx\]

\end{document}

